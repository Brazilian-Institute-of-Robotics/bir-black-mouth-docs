\documentclass[conference]{IEEEtran}
\IEEEoverridecommandlockouts
% The preceding line is only needed to identify funding in the first footnote. If that is unneeded, please comment it out.
\usepackage{cite}
\usepackage{amsmath,amssymb,amsfonts}
\usepackage{algorithmic}
\usepackage{graphicx}
\usepackage{textcomp}
\usepackage{xcolor}
\usepackage{caption}
\usepackage{subcaption}
\usepackage{subfiles}
\usepackage{siunitx}
\usepackage{gensymb}

\graphicspath{{\subfix{../images/}}}
\def\BibTeX{{\rm B\kern-.05em{\sc i\kern-.025em b}\kern-.08em
    T\kern-.1667em\lower.7ex\hbox{E}\kern-.125emX}}
\begin{document}

\title{Conference Paper Title*\\
{\footnotesize \textsuperscript{*}Note: Sub-titles are not captured in Xplore and
should not be used}
\thanks{Identify applicable funding agency here. If none, delete this.}
}

\author{\IEEEauthorblockN{1\textsuperscript{st} Given Name Surname}
  \IEEEauthorblockA{\textit{dept. name of organization (of Aff.)} \\
    \textit{name of organization (of Aff.)}\\
    City, Country \\
    email address or ORCID}
  \and
  \IEEEauthorblockN{2\textsuperscript{nd} Given Name Surname}
  \IEEEauthorblockA{\textit{dept. name of organization (of Aff.)} \\
    \textit{name of organization (of Aff.)}\\
    City, Country \\
    email address or ORCID}
  \and
  \IEEEauthorblockN{3\textsuperscript{rd} Given Name Surname}
  \IEEEauthorblockA{\textit{dept. name of organization (of Aff.)} \\
    \textit{name of organization (of Aff.)}\\
    City, Country \\
    email address or ORCID}
}

\maketitle

\begin{abstract}
  With the growth of robotics, mobile robots are increasingly becoming a common aspect in key sectors of the economy. When compared to wheeled and tracked robots, legged robots present greater mobility and maneuverability, but less locomotion stability, which in turn demands more complex control systems. This paper aims to develop a quadruped robot capable of walking when teleoperated and assess its locomotion performance based on experiments. The first part of the methodology was dedicated to research and the definition of the robot's concept. The following stage involved the development of the robot's software, simulation, and, in parallel, the design, fabrication and assembly of the prototype. After both stages were done, the software was integrated into the prototype. After the integration stage, the locomotion experiments were conducted. Based on these experiments, it was possible to conclude that the robot was able to perform gaits of different sizes and walk in both plane and irregular terrains. Furthermore, it was observed that the body's rotation PID controller contributed to reducing the oscillation of the robot's body in both roll and pitch during walking in both terrains.
\end{abstract}

\begin{IEEEkeywords}
  Quadruped robot; Locomotion; Control.
\end{IEEEkeywords}

\section{Introduction}
Com o avanço da robótica, robôs móveis estão cada vez mais ganhando espaço em setores chave da economia como o comercial, industrial e militar. Os robôs terrestres que se locomovem com pernas têm se mostrado mais eficientes para se locomover em terrenos irregulares, inclinados e escorregadios, e também para superar obstáculos \cite{X.134}. Quando comparados com robôs com rodas, robôs com pernas ainda possuem melhor mobilidade e manobrabilidade em ambientes complexos, o que possibilita transitar por caminhos não necessariamente contínuos. Entre os robôs com pernas, os quadrúpedes vêm ganhando destaque por apresentarem maior estabilidade e uma estrutura mais simples quando comparados aos bípedes e hexápodes \cite{Shi2021}.

O uso de uma plataforma com quatro pernas requer um sistema de locomoção robusto que envolve o controle de equilíbrio da plataforma, o controle das juntas do robô e o planejamento de marchas. Para poder aplicar todos esses conceitos e desenvolver aplicações reais com robôs quadrúpedes, é fundamental estudar sua estrutura física, os tipos de marcha que eles podem realizar e os métodos de controle de locomoção que são utilizados nesses robôs.

O objetivo deste trabalho é desenvolver um sistema robótico do tipo quadrúpede capaz de se locomover de forma estável quando teleoperado. Serão estudados aspectos construtivos, de locomoção e de controle desse tipo robô. Além disso, será projetado um protótipo que será simulado e fabricado, a fim de testar os algoritmos de locomoção não apenas em um ambiente virtual, mas também na prática. Por fim, serão realizados experimentos de performance de locomoção com o protótipo físico, os quais analisarão a estabilidade do robô em ambientes de terreno plano e irregular.

A seção \ref{sec:the_quadruped_robot} tratará dos conceitos estudados para o desenvolvimento do robô. A seção \ref{sec:methodology} explicará a metodologia utilizada durante o projeto. Na seção \ref{sec:the_caramel}, será apresentado o Caramelo, o robô desenvolvido para essa pesquisa. A seção \ref{sec:results} apresentará os resultados dos testes e experimentos realizados, ao passo que a seção \ref{sec:conclusion}, as conclusões feitas a partir dos resultados. Por fim, a seção \ref{sec:final_points} apresentará as considerações finais e as sugestões dos autores para trabalhos futuros.


\section{O Robô Quadrúpede}

O robô quadrúpede é um sistema robótico móvel que se locomove com a ajuda de pernas. Quando comparado a robôs terrestres que utilizam outros meios de locomoção --- como rodas ou esteiras ---, ele apresenta diversas particularidades que lhes confere muitas vantagens quanto à mobilidade, robustez a diferentes terrenos e superação de obstáculos \cite{Biswal2021}. Robôs com rodas e esteiras conseguem navegar pelo espaço, desde que haja um caminho contínuo entre os pontos de origem e de destino. Robôs com pernas, por outro lado, são capazes de escolher os melhores pontos no terreno para apoiar suas patas, o que permite uma navegação em caminhos discretos (com obstáculos de grande inclinação e variação de altura) \cite{Yao2021}. Essa capacidade de se adaptar a terrenos desnivelados favorece sua aplicação em setores  industrial e militar, atuando em missões de inspeção e de resgate. Por outro lado, esse tipo de robô apresenta menor estabilidade de locomoção e, por consequência, maior complexidade de controle.

Robôs com pernas também apresentam diversas diferenças entre si, majoritariamente ligadas à quantidade de pernas que possuem. A quantidade de pernas de um robô está diretamente relacionada a sua estabilidade, capacidade de locomoção e eficiência. Os bípedes possuem baixa estabilidade, visto que se apoiam em apenas uma perna para poderem andar. Aqueles com múltiplas pernas (mais de quatro) possuem maior estabilidade, visto que conseguem manter pelo menos três pontos de apoio no solo enquanto realizam um passo. No entanto, cada perna representa um conjunto adicional de juntas e atuadores, diminuindo a eficiência do sistema como um todo. Os quadrúpedes conseguem unir vantagens desses dois tipos ao apresentar um balanço entre estabilidade e eficiência. Eles possuem uma estabilidade passiva quando estáticos, pois se apoiam em quatro pontos. Além disso, também são capazes de navegar de forma estável em baixas velocidades, movendo uma perna por vez enquanto as outras três permanecem no solo. Isso elimina a redundância presente nos robôs com múltiplas pernas, aumentando sua eficiência \cite{Yao2021}.

\subsection{Estrutura e design}
Pelo fato de robôs quadrúpedes terem se tornado um grande foco de pesquisa nos últimos anos, diferentes \textit{designs} já foram pesquisados. Esses \textit{designs} se distinguem quanto à estrutura, configuração de pernas e o número de graus de liberdade (GDL) por perna.

Um dos tipos de estrutura existente é a de tipo mamífero. A estrutura tipo mamífero tem esse nome devido a sua semelhança com a postura de quadrúpedes, como cachorros e cavalos. Kitano \textit{et al.}, em \cite{Kitano2016}, analisa dois tipos diferentes de estruturas de robôs quadrúpedes: a do tipo mamífero (figura \ref{fig:robots_structures_b}) e a do tipo \textit{sprawling} (figura \ref{fig:robots_structures_c}). Segundo sua análise, a primeira permite alcançar maiores velocidades por possuir duas juntas no plano sagital. Além disso, ela também apresenta maior eficiência, pois os atuadores requerem menos torque para sustentar o robô: sua estrutura mais compacta diminui o braço de alavanca sobre o qual a força peso do robô atua. Essa estrutura também favorece a navegação em ambientes estreitos, onde um robô do tipo \textit{sprawling}, por exemplo, teria dificuldades de acessar.

\begin{figure}[htbp]
  \centering
  \begin{subfigure}[htbp]{0.23\textwidth}
    \centering
    \includegraphics[width=0.9\textwidth]{Spot.png}
    \caption{Spot.}
    \label{fig:robots_structures_b}
  \end{subfigure}
  \begin{subfigure}[htbp]{0.23\textwidth}
    \centering
    \includegraphics[width=1.0\textwidth]{Titan.png}
    \caption{TITAN-XIII.}
    \label{fig:robots_structures_c}
  \end{subfigure}
  \centering
  \caption{Exemplos de robôs com estruturas do tipo mamífero e \textit{sprawling}.}
  Fonte: Adaptado de  \cite{Kitano2016} e \cite{SpotImg1}.
  \label{fig:robots_structures}
\end{figure}

Os robôs quadrúpedes que utilizam essa estrutura também se diferenciam quanto à configuração das pernas. As duas configurações mais utilizadas podem ser vistas na figura \ref{fig:joint_configurations}. Robôs como o \textit{Spot}, \textit{MIT Cheetah} e \textit{Stanford Pupper} utilizam a configuração \textit{full-elbow}, enquanto outros como o \textit{ANYmal}, \textit{StarlETH} e \textit{BigDog} adotam a configuração \textit{elbow-knee}. Yao \textit{et al.}, em \cite{Yao2021}, acreditam que a configuração \textit{elbow-knee} possibilita maior estabilidade, mas as características de movimento da configuração \textit{full-elbow} podem ser superiores.

\begin{figure}[htbp]
  \centering
  \begin{subfigure}[htbp]{0.24\textwidth}
    \centering
    \includegraphics[width=1.0\textwidth]{full_elbow.png}
    \caption{\textit{full-elbow}.}
    \label{fig:joint_configurations_a}
  \end{subfigure}
  \begin{subfigure}[htbp]{0.24\textwidth}
    \centering
    \includegraphics[width=1.0\textwidth]{elbow_knee.png}
    \caption{\textit{elbow-knee}.}
    \label{fig:joint_configurations_d}
  \end{subfigure}

  \caption{Tipos de configuração de pernas para robôs com estrutura tipo mamífero.}
  Fonte: Adaptado de \cite{Yao2021}.
  \label{fig:joint_configurations}
\end{figure}

O número de juntas nas pernas, que coincide com a quantidade de GDL do robô, também é um dos aspectos estudados sobre os quadrúpedes. A maioria apresenta 3 GDL por perna, o que é suficiente para que o robô consiga mover suas patas em três dimensões e realize diversos tipos de marchas. A fim de simplificar a estrutura e consequentemente o controle, alguns robôs utilizam apenas 2 GDL por perna, eliminando a junta no corpo que movimenta a perna no plano frontal. Outros robôs buscam performances mais semelhantes ao andar de animais reais, o que demanda maior flexibilidade de movimento, justificando o acréscimo de uma quarta junta. No entanto, como já mencionado, esse acréscimo aumenta a complexidade do controle e prejudica a eficiência. Essa perda de eficiência se dá porque a maior quantidade de atuadores resulta em um maior consumo de energia e em uma maior quantidade de massa.

A massa do robô quadrúpede deve ser a menor possível. Quanto mais leve for o sistema, menos torque será demandado dos motores e, consequentemente, maior sua eficiência. Além disso, a distribuição de massa do robô também é um aspecto muito importante. A massa deve ser localizada majoritariamente no corpo, enquanto as pernas devem possuir baixa inércia. Isso permite que elas se movam rapidamente sem alterar, de forma significativa, o centro de gravidade do robô, o que aumenta a estabilidade e requer menos complexidade de controle. Possuir baixa inércia significa possuir baixa massa. Por outro lado, as pernas devem ser resistentes o suficiente para suportar o peso do robô, além dos distúrbios causados pelo impacto das patas com o chão, o que pode demandar um aumento de massa nas pernas. Portanto, um equilíbrio entre massa e resistência deve ser buscado ao mesmo tempo em que deve-se buscar diminuir a massa total do sistema \cite{Zhong2019}.

\subsection{Movimentação por marchas}
Robôs quadrúpedes se movimentam conforme uma sequência de movimentos coordenados de suas pernas que compõem uma marcha. Uma marcha é definida pelo tempo e local de colocação e levantamento de cada pata, coordenado com o movimento do corpo em seus seis graus de liberdade, para mover o corpo de um lugar para outro  \cite{Song1989}.

A marcha é um aspecto fundamental para garantir que um robô com pernas caminhe de forma eficiente e estável, especialmente em terrenos irregulares \cite{X.129}. Para tanto, é necessário levar em consideração suas etapas e, consequentemente, seu tipo.

Marchas são divididas em duas etapas: \textit{stance} e \textit{swing}. Durante a fase \textit{stance}, as pernas estão no solo e impulsionam o robô para frente. Na etapa de \textit{swing}, as pernas são erguidas para deslocar a pata até o próximo ponto de apoio. É importante ressaltar que as fases de \textit{stance} e \textit{swing} não ocorrem em todas as pernas simultaneamente. A depender do tipo de marcha, algumas pernas podem estar em \textit{swing} enquanto outras estarão em \textit{stance}.

O \textit{trot} é um tipo de marcha muito utilizado por robôs quadrúpedes, devido a sua simplicidade e eficiência. Este tipo de marcha é periódico e simétrico. Marchas periódicas são caracterizadas pela repetição contínua dos mesmos movimentos nos mesmos instantes, dentro de um ciclo de locomoção \cite{de2006quadrupedal}. Já a simetria é uma característica de marchas que movimentam um par de pernas em conjunto, saindo e voltando para o solo de forma sincronizada. No \textit{trot}, as pernas diagonais se movimentam em pares, e quando um par está na etapa de \textit{swing}, o outro está na etapa de \textit{stance}. Outra característica da marcha \textit{trot} é que ela pode ser contínua ou descontínua. Uma marcha contínua mantém o corpo do robô em movimento constante, enquanto que a descontínua submete o corpo a um movimento intermitente \cite{de2006quadrupedal}. Portanto, quando a marcha \textit{trot} é contínua, as pernas em \textit{stance}, além de sustentarem o robô, deslocam o corpo na direção do movimento, o que exige maior capacidade de controle. Em contrapartida, quando ela é descontínua, o corpo fica estático, esperando as pernas em \textit{swing} terminarem o movimento para, então, ser deslocado, tão logo as quatro pernas estejam no solo.

\subsection{Controle de locomoção}

Todo o controle de locomoção do robô é realizado pelo planejador de marchas. Ele é o responsável por enviar os comandos para que as pernas se movam para os locais desejados, no momento esperado. Logo, o planejador irá apenas ditar o ponto no espaço no qual cada pata do robô deve estar, em relação a um eixo de referência, cabendo ao sistema de controle executar o movimento. A seguir, serão discutidos dois itens fundamentais do sistema de controle de um robô quadrúpede: o modelo cinemático e as estratégias de controle.

\subsubsection{Modelo cinemático}
O modelo cinemático de um robô quadrúpede descreve a relação entre a posição de uma pata em três dimensões com a rotação de cada junta da sua respectiva perna. Como todas as pernas do robô são iguais, pode-se formular as relações de apenas uma perna e replicá-la quatro vezes, acrescentando as devidas translações e rotações, para obter o modelo cinemático de todo o sistema.

O modelo cinemático pode ser usado para resolver dois problemas: a cinemática direta e a cinemática inversa. A cinemática direta fornece a posição de uma pata em $(x, y, z)$ em função dos ângulos das juntas, enquanto que a cinemática inversa fornece os ângulos das juntas correspondentes a uma posição da pata no espaço tridimensional. Esses dois problemas são complementares, sendo a saída de um a entrada do outro, e vice-versa. A partir da cinemática inversa, o robô consegue determinar quanto deve rotacionar seus atuadores para mover a pata a alguma distância nas direções $(x, y, z)$. Com a cinemática direta, é possível saber se a pata de fato chegou na posição em que ele deveria estar. Portanto, ambos são muito importantes para o controle de locomoção do robô.

\subsubsection{Estratégias de controle}
A locomoção de robôs quadrúpedes, em geral, segue uma sequência de passos. Raibert propõe em \cite{Raibert1986} um método de controle baseado em três etapas: controle de salto, controle de velocidade e controle de postura do corpo. Essa estratégia de controle foi utilizada para controlar robôs com uma, duas, e quatro pernas (o controle de salto se justifica porque robôs com apenas uma perna só podem ser locomover saltando). Sua premissa básica era a de que apenas uma perna estaria em \textit{stance} ou em \textit{swing} por vez. Para que robôs com mais de duas pernas satisfaçam essa premissa, foi proposto o conceito de pernas virtuais. Isto é, um conjunto de pernas deve realizar igual comportamento quando em \textit{swing} e \textit{stance} e as fases de \textit{swing} e \textit{stance} de cada conjunto devem ser alternadas. Esse conceito foi utilizado para embasar o uso de marchas simétricas e periódicas como o \textit{trot}.

Essa estratégia de controle em três etapas foi responsável por locomover robôs com pernas rígidas (apenas 2 GDL por perna, sendo uma junta rotativa e outra prismática) de maneira simples. No entanto, esses robôs apenas operavam no terreno plano e controlado do laboratório. A fim de possibilitar a operação de robôs com pernas em terrenos desnivelados e de difícil mobilidade, Raibert \textit{et al.} propôs um outro sistema de controle no seu trabalho sobre o \textit{BigDog} \cite{RAIBERT200810822}. O \textit{BigDog} é um robô quadrúpede com 4 GDL por perna movido por atuadores hidráulicos. Essa maior flexibilidade de movimentação das pernas permite controlar a locomoção do robô sem que este precise saltar, sendo possível, então, dividir o controle de locomoção em duas etapas principais: controle de \textit{stance} e controle de \textit{swing}. Como o nome sugere, o controlador de \textit{stance} é o responsável por controlar o comportamento das pernas na fase de \textit{stance}, enquanto o controlador de \textit{swing} é o responsável por controlar as pernas em \textit{swing}. Vale lembrar que durante a marcha, algumas pernas podem estar na etapa de \textit{swing} enquanto outras estão na etapa de \textit{stance}, o que significa que esses controladores ora assumem o comando de um par de pernas, ora de outro (essa troca não necessariamente se dá em pares, porém isso é válido para marchas simétricas como o \textit{trot}). Quem define o momento em que cada controlador assume o controle de uma determinada perna é o planejador de marchas.

O modo como cada perna se comporta durante as etapas de \textit{stance} e \textit{swing} pode variar em diversos aspectos, mas ainda é possível elencar semelhanças gerais. No início da etapa de \textit{swing}, calcula-se o local do próximo ponto de apoio das patas com base na velocidade desejada para o robô e uma trajetória de passo até esse ponto. Essa trajetória pode ter o formato de uma curva senoidal \cite{X.118}, triangular \cite{StanfordPupper}, de Bezier \cite{HackadayQuadruped}, cicloidal \cite{Shi2021} \cite{X.58}, entre outras. Uma consideração que pode ser feita a fim de simplificar o sistema de controle é de que a movimentação das pernas durante a fase de \textit{swing} não interfere no movimento do corpo do robô. Para que isto seja válido, é necessário controlar a força com que a pata toca o solo, a fim de diminuir os distúrbios causados no corpo do robô. A força de contato entre as patas e o solo é um ponto-chave para a estabilidade do quadrúpede \cite{X.118}. Nesse sentido, a trajetória cicloidal ganha destaque por conta de sua primeira derivada nula no momento em que se aproxima do seu ponto mínimo.

Já na fase de \textit{stance}, as pernas devem manter o robô em equilíbrio, além de deslocar o corpo na direção desejada de locomoção. Para isso, alguns robôs utilizam trajetórias para as patas com um formato pré-determinado, assim como na fase de \textit{swing}, não necessariamente repetindo o mesmo formato de curva \cite{X.118, X.58}. Além disso, controladores de equilíbrio também podem ser implementados nessa etapa. Esses controladores visam estabilizar os ângulos de \textit{pitch} e \textit{roll} do robô \cite{Shi2021, StanfordPupper, HackadayQuadruped, Notspot} ou ainda outros graus de liberdade \cite{X.134, Chen2020140736, Zhang2016284}. Eles podem controlar diretamente a angulação do corpo do robô (com o auxílio de um sensor inercial) e/ou a força de contato com o solo em cada perna, por exemplo. No entanto, alguns trabalhos se baseiam apenas no controle individual de cada junta para manter o robô em equilíbrio, o que é uma abordagem mais simples, mas que pode falhar, especialmente em terrenos irregulares.

\section{O Robô Caramelo}

Baseado nos conceitos apresentados, foi projetado, simulado e desenvolvido um robô quadrúpede nomeado de Caramelo (figuras \ref{fig:terrenos}, \ref{fig:caramel_body} e \ref{fig:moving_body}). Caramelo é um robô quadrúpede de pequeno porte voltado para pesquisa e educação. Seu \textit{hardware} foi modelado inteiramente pela equipe e impresso com impressora 3D no material ABS. Os atuadores do robô são servomotores do modelo \textit{dynamixel} MX-28 e sua central de processamento é composta por uma \textit{RaspberryPi} 4. Além disso, ele conta com um sensor inercial modelo MPU6050, que está instalado no corpo do robô. Este sensor contém um giroscópio e um acelerômetro, o que permite obter a aceleração linear, a velocidade angular e a orientação do corpo do robô. Todo o \textit{software} foi desenvolvido com o \textit{Robot Operating System 2 Humble} (\textit{ROS2}) \cite{ROS2Humble}, que consiste em um \textit{framework} de robótica \textit{open source} com vários recursos disponíveis para facilitar o desenvolvimento de sistemas robóticos.

A estrutura do Caramelo é do tipo mamífero e a configuração das pernas é a \textit{full-elbow}. Além disso, possui 3 GDL por perna, o que permite uma grande liberdade de movimentação para as patas. Seu \textit{design} foi pensado para favorecer o balanço de massas entre o corpo e as pernas do robô, ou seja, a maior parte da massa se encontra no corpo ou próxima a ele. Os componentes eletrônicos internos, que abrangem sensores, unidades de processamento e a interface de comunicação com os atuadores, foram dispostos de forma simétrica, a fim de manter o centro de massa o mais próximo do centro do corpo. Os motores (componentes que contribuem com a maior massa para o sistema) foram dispostos o mais próximo possível do corpo. Um destaque para o motor que atua na junta da tíbia foi sua instalação na parte superior do fêmur, com o objetivo de diminuir o momento de inércia da perna. Essa escolha demandou a adição de um sistema de transmissão entre o eixo do motor e a tíbia, formado por uma haste rígida de metal com uma junta esférica em cada extremidade.

A locomoção do Caramelo foi desenvolvida baseada nas marchas periódicas e simétricas. Dessa forma, foi adotada a marcha \textit{trot} como a marcha principal do robô. Embora sua estrutura permita a realização de muitos outros tipos de marcha, neste trabalho, foi considerada apenas o \textit{trot}, devido a sua simplicidade e eficiência. Com o objetivo de diminuir a complexidade do controle de locomoção, foi adotada uma marcha descontínua, ou seja, o corpo do robô se desloca apenas quando todas as patas estão no solo. A sequência de etapas da marcha do Caramelo pode ser vista na figura \ref{fig:trot_pattern}, cujas áreas em branco representam a etapa de \textit{swing} e as em cinza a de \textit{stance}. É possível perceber que sempre o mesmo par de pernas diagonais se move no mesmo instante. Entre duas etapas consecutivas de \textit{swing}, há um momento em que todas as patas estão em \textit{stance}, que é quanto o corpo do robô é deslocado no sentido desejado de locomoção.

\begin{figure}[htbp]
  \centering
  \includegraphics[width=0.45\textwidth]{trot_pattern.png}
  \vfill
  \caption{Padrão de movimentação da marcha para cada perna.}
  Fonte: autores.
  \label{fig:trot_pattern}
\end{figure}

O sistema de controle do robô é composto por dois subsistemas principais: os controladores individuais de cada junta e os controladores da angulação do corpo do robô. Os controladores das juntas são os próprios controladores PID embarcados nos motores \textit{dynamixel}. Foi utilizada a interface de controle de posição com o atuador, de forma que o \textit{setpoint} de controle enviado para cada motor é o ângulo em radianos para o qual ele deve rotacionar. O modelo cinemático do robô, apresentado na seção \ref{sec:detail_inv_kinematics}, é o responsável por mapear não apenas a posição tridimensional de cada pata com a angulação de cada junta, mas também a posição do corpo em seis dimensões: translação e rotação em $(x, y, z)$. Dessa forma, é possível controlar cada pata e o corpo do robô ao mesmo tempo de forma independente. Os controladores de angulação do corpo são dois controladores PID em paralelo, responsáveis por controlar o ângulo de \textit{pitch} (rotação em $y$) e o de \textit{roll} (rotação em $x$).

O planejador de marchas é o responsável por controlar cada pata do robô e, por consequência, o corpo. Ele calcula a trajetória que cada pata deve realizar, com base nas etapas de \textit{stance} e \textit{swing}, e envia o próximo ponto em que cada pata deve estar, a uma frequência de $\SI{50}{\hertz}$. Essa frequência foi definida por ser a maior que o sistema de processamento do robô foi capaz de suportar e por respeitar o teorema Nyquist, que indica que, nesse caso, a frequência deve ser maior do que $\SI{4}{\hertz}$ --- já que o período do passo é $0.5 s$. Além disso, o planejador de marchas também considera o esforço de controle enviado pelos controladores de angulação, de modo a manter o corpo do robô em $0^{\circ}$ a todo momento.

A seguir, será apresentado o desenvolvimento do modelo cinemático, dos controladores de angulação e da trajetória que cada pata realiza na etapa de \textit{swing}.

\subsection{Modelo cinemático do Caramelo}
\label{sec:detail_inv_kinematics}

Como dito anteriormente, o modelo cinemático é utilizado para resolver a cinemática inversa e a cinemática direta do robô. Para a cinemática direta, foi utilizado o pacote \textit{tf2}, um recurso disponível no \textit{ROS2} que facilita o gerenciamento de transformações entre eixos de coordenadas. A cinemática inversa, por outro lado, foi feita com base em uma análise geométrica. As variáveis $\theta_1$, $\theta_2$ e $\theta_3$ expressam a posição angular de cada uma das juntas de uma perna do robô e são calculadas com auxílio das equações \ref{eq:theta1} a \ref{eq:B} em função da posição $(x_{IK}, y_{IK}, z_{IK})$ desejada para a pata e dos comprimentos $L_1$, $L_2$ e $L_3$ (figura \ref{fig:caramel_tfs}).
\begin{equation}
  \label{eq:theta1}
  \theta_1 = \arctan{(\frac{x_{IK}}{y_{IK}})} - \arctan{(\frac{L_1}{a})}
\end{equation}
\begin{equation}
  \label{eq:theta2}
  \theta_2 = \frac{\pi}{2} - \arctan{(\frac{a}{z_{IK}}}) - \arctan{(\frac{\sqrt{1-A^2}}{A})}
\end{equation}
\begin{equation}
  \label{eq:theta3}
  \theta_3 = \arctan(\frac{\sqrt{1-B^2}}{B})
\end{equation}
\begin{equation}
  \label{eq:a}
  a = \sqrt{x_{IK}^2+y_{IK}^2-L_1^2}
\end{equation}
\begin{equation}
  \label{eq:A}
  A =\frac{a^2+z^2+L_2^2-L_3^2}{2L_2\sqrt{a^2+z_{IK}^2}}
\end{equation}
\begin{equation}
  \label{eq:B}
  B = \frac{a^2+z_{IK}^2-L_2^2-L_3^2}{2L_2L_3}
\end{equation}

\begin{figure}[htbp]
  \centering
  \includegraphics[width=0.4\textwidth]{caramel_tfs.png}

  \caption{Links da perna do robô.}
  Fonte: autores.
  \label{fig:caramel_tfs}
\end{figure}

Essas equações são úteis para o cálculo da posição de uma única perna, mas são insuficientes para realizar a cinemática do corpo do robô. Desta forma, um \textit{frame} central, chamado de \textit{base\_link} (figura \ref{fig:caramel_body}), é utilizado como referência, e uma matriz $T_M$ (eqs. \ref{eq:Tm} e \ref{eq:Rxyz}) é utilizada para realizar a cinemática do corpo, a partir das translações $(x_c, y_c, z_c)$ e rotações $(\alpha, \beta, \gamma)$ desejadas, sendo possível controlar cada um dos 6 graus de liberdade. Para tanto, as transformações $T_{FR}$, $T_{FL}$, $T_{BL}$ e $T_{BR}$ de cada um dos ombros (\textit{hip\_links}) em relação ao \textit{base\_link} são necessárias.
\begin{equation}
  \label{eq:Tm}
  T_M =
  \begin{bmatrix}
      &         &   & x_c \\
      & R_{xyz} &   & y_c \\
      &         &   & z_c \\
    0 & 0       & 0 & 1
  \end{bmatrix}
\end{equation}
\begin{equation}
  \label{eq:Rxyz}
  \begin{split}
    R_{xyz} =
    \begin{bmatrix}
      1 & 0          & 0           \\
      0 & \cos\alpha & -\sin\alpha \\
      0 & \sin\alpha & \cos\alpha
    \end{bmatrix}
    \\.
    \begin{bmatrix}
      \cos\beta  & 0 & \sin\beta \\
      0          & 1 & 0         \\
      -\sin\beta & 0 & \cos\beta
    \end{bmatrix}
    \\.
    \begin{bmatrix}
      \cos\gamma & -\sin\gamma & 0 \\
      \sin\gamma & \cos\gamma  & 0 \\
      0          & 0           & 1
    \end{bmatrix}
  \end{split}
\end{equation}

\begin{figure}[htbp]
  \centering
  \vspace{-0.75cm}
  \includegraphics[width=0.35\textwidth]{caramel_body.drawio.png}

  \caption{Eixos do robô em posição de repouso.}
  Fonte: autores.
  \label{fig:caramel_body}
\end{figure}

O cálculo das angulações de cada perna então é feito utilizando como entrada os valores $(x_{IK}, y_{IK}, z_{IK})$ resultantes de cada uma das transformações, conforme a equação \ref{eq:xyzik}. O mesmo cálculo é feito para as demais pernas, utilizando  $T_{FL}$, $T_{BL}$ e $T_{BR}$.
\begin{equation}
  \label{eq:xyzik}
  \begin{bmatrix}
    x_{IK} \\
    y_{IK} \\
    z_{IK} \\
    1
  \end{bmatrix}= (T_M.T_{FR})^{-1}.
  \begin{bmatrix}
    x \\
    y \\
    z \\
    1
  \end{bmatrix}
\end{equation}

Desta forma, a cinemática inversa é capaz de computar as angulações  $\theta_1$, $\theta_2$ e $\theta_3$ de cada uma das pernas a partir da posição $(x, y, z)$ das patas em relação ao link central do robô e às translações $(x_c, y_c, z_c)$ e rotações $(\alpha, \beta, \gamma)$ desejadas para o corpo. Entretanto, em muitos casos, é mais conveniente realizar o cálculo dos ângulos passando como entrada as posições $(x, y, z)$ das patas em relação à sua posição \textit{default}, ou seja, a posição do seu \textit{foot\_link} quando o robô está em seu estado de repouso (figura \ref{fig:caramel_body}). Para isso, é possível realizar, previamente ao cálculo das angulações, mais uma transformação, desta vez do \textit{base\_link} para cada uma das posições \textit{default} das patas (equação \ref{eq:xyzik_foot}).
\begin{equation}
  \label{eq:xyzik_foot}
  \begin{bmatrix}
    x_{ik} \\
    y_{ik} \\
    z_{ik} \\
    1
  \end{bmatrix}= (T_M.T_{FR})^{-1}.
  (F_{FR}.
  \begin{bmatrix}
    x \\
    y \\
    z \\
    1
  \end{bmatrix})
\end{equation}

\subsection{Controle de angulação}

Os controladores de angulação são dois controladores PID em paralelo, responsáveis por controlar a rotação de \textit{roll} e \textit{pitch} do corpo do robô. Eles atuam de forma independente, controlando a rotação do corpo em ambos os eixos simultaneamente. Ambos os controladores são iguais e foram implementados seguindo o modelo apresentado no diagrama de blocos da figura \ref{fig:pid}.
\begin{figure}[htbp]
  \centering
  \includegraphics[width=0.48\textwidth]{PID.drawio.png}
  \caption{Controlador PID projetado.}
  Fonte: autores.
  \label{fig:pid}
\end{figure}

O IMU é o sensor responsável por medir a rotação do corpo do robô, possibilitando a realimentação das saídas do sistema. O limitador foi adicionado para evitar que sejam enviados valores que extrapolam os limites de rotação das juntas do robô. Os esforços de controle são enviados para o planejador de marchas que, por sua vez, envia os comandos de movimentação para os controladores das juntas.

\subsection{Planejador de trajetória}

O planejador de trajetória é responsável por calcular a trajetória que cada pata deve realizar, tanto na fase de \textit{stance} quanto na fase de \textit{swing}. Para o Caramelo, a trajetória é uma curva cicloidal em ambas as etapas. Como apresentado em \cite{Shi2021}, uma curva cicloidal pode ser definida no espaço tridimensional entre os pontos $(x_o, y_o, z_o)$ e $(x_f, y_f, z_f)$ em função do tempo $t$ pelas equações (\ref{eq:traj_x}) a (\ref{eq:traj_k}), sendo $H$ a altura do passo e $T$ o período.
\begin{equation}
  x = (x_f - x_o) \frac{K - \sin{(K)}}{2 \pi} + x_o
  \label{eq:traj_x}
\end{equation}
\begin{equation}
  y = (y_f - y_o) \frac{K - \sin{(K)}}{2 \pi} + y_o
  \label{eq:traj_y}
\end{equation}
\begin{equation}
  z = H \frac{1 - \cos{(K)}}{2} + z_o
  \label{eq:traj_z}
\end{equation}
\begin{equation}
  K = \frac{2 \pi t}{T}
  \label{eq:traj_k}
\end{equation}

O gráfico de uma curva cicloidal no espaço 3D pode ser vista na figura \ref{fig:traj_space}. A mesma trajetória é ilustrada na figura \ref{fig:traj_time} em função do tempo. É possível perceber que a curva possui a primeira derivada nula no momento em que a pata toca o solo, o que é favorável ao controle de malha aberta, uma vez que quanto mais suave a aterrissagem, menos distúrbios são causados no sistema.

\begin{figure*}[h]
  \centering
  \begin{subfigure}[t]{0.32\textwidth}
    \centering
    \includegraphics[width=1.0\textwidth]{Cycloid_space.png}
    \caption{Curva no espaço 3D.}
    \label{fig:traj_space}
  \end{subfigure}
  \begin{subfigure}[t]{0.32\textwidth}
    \centering
    \includegraphics[width=1.0\textwidth]{Cycloid_time.png}
    \caption{Curva no tempo.}
    \label{fig:traj_time}
  \end{subfigure}
  \begin{subfigure}[t]{0.32\textwidth}
    \centering
    \includegraphics[width=1.0\textwidth]{Cycloid_modified.png}
    \caption{Curva com disposição de pontos modificada.}
    \label{fig:traj_time_modified}
  \end{subfigure}
  \vfill
  \caption{Trajetórias cicloidais para o passo de robô.}
  Fonte: autores.
  \label{fig:traj_curve}
\end{figure*}

Além da altura, distâncias em $x$ e $y$ e período, o planejador de trajetória do Caramelo também conta com dois parâmetros que têm como objetivo melhorar ainda mais o controle da força com que a pata toca o chão. Como os atuadores do robô são servomotores controlados por posição, o torque é proporcional ao deslocamento que este deve realizar entre os pontos da trajetória. Ou seja, quanto maior a resolução da trajetória, mais suave será o movimento. No entanto, a resolução da trajetória $N$ é fixa, dada em função do período do passo e da frequência de controle do planejador de marchas ($\SI{50}{\hertz}$) $N = 50T$. Logo, a estratégia adotada é a de espaçar a mesma quantidade de pontos de forma desigual ao longo do período do passo, de forma que haja mais pontos próximos ao momento em que a pata aterrissa no solo, e menos pontos próximos ao momento em que ela é erguida. O parâmetro $P_T$ é uma fração do período total do passo, e o parâmetro $P_N$, a fração do número total de pontos do passo que deve se encontrar entre o tempo $0$ e $P_T \cdot T$. Em outras palavras, se $P_T = 0,66$ e $P_N = 0,33$, $33\%$ de $N$ estará nos primeiros dois terços do período, enquanto os $67\%$ restantes estarão no um terço final. A trajetória, considerando esses parâmetros, está ilustrada na figura \ref{fig:traj_time_modified}.

\section{Results}

To assess the robot's locomotion performance with the implemented stabilization controller, the robot's capability of following a forward velocity setpoint and the oscillation of its body while walking were evaluated. An experiment was conducted, in which the robot should walk in a straight line of 1.5 meters in both flat and irregular terrain, with and without the stabilization controller, totalizing four tests. The flat terrain consisted of a cement floor and the irregular terrain, a ground formed by small loose stones. Both are illustrated in Figures \ref{fig:terreno_plano} and \ref{fig:terreno_irregular}, respectively. For each test, the robot received a forward velocity command of 0.05 m/s, and its mean velocity, as well as its pitch and roll oscillation, were recorded. The mean velocity was calculated based on the time it took to travel 1.5 meters and the oscillation is the difference between the maximum and minimum readings of the IMU. The calculated velocity was compared to the desired velocity sent to the robot and the stability data was analyzed statistically: first, the Shapiro-Wilk test was applied to the distribution to check whether the distribution was normal; lastly, the results were compared by performing an analysis of variance (ANOVA). For all the tests, the robot's gait settings consisted of a height of 5 cm, a period of 0.5 seconds, a resolution of 25 points, $P_T$ = 0.66, and $P_N$ = 0.33. For each test type, 10 samples were collected.

\begin{figure}[htbp]
  \centering
  \begin{subfigure}[htbp]{0.24\textwidth}
    \centering
    \includegraphics[width=1.0\textwidth]{terreno_plano.jpeg}
    \caption{Flat.}
    \label{fig:terreno_plano}
  \end{subfigure}
  \begin{subfigure}[htbp]{0.24\textwidth}
    \centering
    \includegraphics[width=1.0\textwidth]{terreno_irregular.jpeg}
    \caption{Irregular.}
    \label{fig:terreno_irregular}
  \end{subfigure}
  \vfill
  \caption{Terrains used for the experiment}
  \label{fig:terrenos}
\end{figure}

\begin{table}[!htb]
  \centering
  \begin{tabular}{ccccc}
    \hline
    \textbf{Test}             & \textbf{1}   & \textbf{2}  & \textbf{3}   & \textbf{4}   \\ \hline
    Ter.                       & Flat        & Flat       & Irreg.       & Irreg.       \\ \hline
    \begin{tabular}[c]{@{}c@{}}S. C. \end{tabular} & No          & Yes         & No          & Yes          \\ \hline
    \begin{tabular}[c]{@{}c@{}}Vel. \\ $(cm/s)$ \end{tabular} & 2.15         & 3.68        & 2.03         & 2.39         \\ \hline
    \begin{tabular}[c]{@{}c@{}} $\sigma_{Vel}$  \\ $(cm/s)$ \end{tabular} & 0.040        & 0.075       & 0.016        & 0.047        \\ \hline
    \begin{tabular}[c]{@{}c@{}} $\Delta_{Roll}$ \end{tabular} & 12.59\degree & 8.81\degree & 13.37\degree & 11.13\degree \\\hline
    \begin{tabular}[c]{@{}c@{}} $\sigma_{Roll}$ \end{tabular} & 2.54\degree  & 2.36\degree & 0.78\degree  & 1.47\degree  \\ \hline
    \begin{tabular}[c]{@{}c@{}} $\Delta_{Pitch}$ \end{tabular} & 11.55\degree & 8.31\degree & 15.19\degree & 12.09\degree \\ \hline
    \begin{tabular}[c]{@{}c@{}} $\sigma_{Pitch}$ \end{tabular} & 1.89\degree  & 2.63\degree & 1.44\degree  & 1.66\degree  \\ \hline
  \end{tabular}

  \caption{Velocity and stability results}
  \label{tab:vel_stab}
\end{table}

Table \ref{tab:vel_stab} summarizes the collected results. The robot did not reach a velocity of 0.05 m/s in any test. This result is due to errors in the final position of each leg during the swing phase: the legs could not reach the desired $(x,y)$ position required to maintain the desired velocity. The fact that the final position of one foot is a function of its motors' final orientation, indicates that the motors did not reach their desired orientation. Before the experiment, the gait movement was tested with the legs off the ground (almost no load), and no significant errors were noted. Since the errors only appeared while walking, it is perhaps caused by the fact that the motors were not powerful enough to support the robot's weight appropriately while walking.

In addition to the velocity results, the pitch and roll oscillations were compared to determine whether or not the stabilization controller lowered the robot's body oscillation. The Shapiro-Wilk test was applied to the data from all the tests, after removing outliers, and every test returned a p-value > 0.05, which indicates that all the distributions followed a normal curve. The ANOVA was, then, applied to compare the roll and pitch oscillations from tests 1 and 2. The same process was applied to the tests 3 and 4 as well. The ANOVA revealed that the roll and pitch distributions of the tests with and without stabilization control were significantly different. Table \ref{tab:vel_stab} shows that the mean value of the tests with stabilization control was less than the mean value of the test without it, which proves that it indeed lowered the robot's body oscillations in both terrain types. Figure \ref{fig:imu_test} better pictures the data distribution. Despite the more scattered distribution presented, 3 out of 4 of the tests with stabilization control had its third quartile less than the first quartile of the distribution without the controller for the same terrain type, demonstrating that the controller lowered the oscillations in $75\%$ of collected test samples.

\section{Conclusion}

This paper addressed the design, prototyping and testing of a quadruped robot. An experiment was held to assess the robot's walking performance. The results showed that the robot did not reach the desired velocity in any of the tests. This was due to errors during the gait's swing phase, likely caused by the external disturbances caused by the robot's weight. These disturbances were not compensated, because the gait trajectory tracking is open-loop (only the individual joint's controllers are closed-loop). In addition to that, it was possible to conclude that the implemented stabilization controller was able to lower the robot's body oscillations while walking in pitch and roll in both flat and irregular terrains.

Reviewing the robot's actuators is recommended for future work. Perhaps a model with more torque is required for more precise control performance.

The robot's project is open source and is available on GitHub \cite{CaramelRepo}.

\section*{Acknowledgment}

The preferred spelling of the word ``acknowledgment'' in America is without
an ``e'' after the ``g''. Avoid the stilted expression ``one of us (R. B.
G.) thanks $\ldots$''. Instead, try ``R. B. G. thanks$\ldots$''. Put sponsor
acknowledgments in the unnumbered footnote on the first page.

\section*{References}

Please number citations consecutively within brackets \cite{b1}. The
sentence punctuation follows the bracket \cite{b2}. Refer simply to the reference
number, as in \cite{b3}---do not use ``Ref. \cite{b3}'' or ``reference \cite{b3}'' except at
the beginning of a sentence: ``Reference \cite{b3} was the first $\ldots$''

Number footnotes separately in superscripts. Place the actual footnote at
the bottom of the column in which it was cited. Do not put footnotes in the
abstract or reference list. Use letters for table footnotes.

Unless there are six authors or more give all authors' names; do not use
``et al.''. Papers that have not been published, even if they have been
submitted for publication, should be cited as ``unpublished'' \cite{b4}. Papers
that have been accepted for publication should be cited as ``in press'' \cite{b5}.
Capitalize only the first word in a paper title, except for proper nouns and
element symbols.

For papers published in translation journals, please give the English
citation first, followed by the original foreign-language citation \cite{b6}.

\begin{thebibliography}{00}
  \bibitem{b1} G. Eason, B. Noble, and I. N. Sneddon, ``On certain integrals of Lipschitz-Hankel type involving products of Bessel functions,'' Phil. Trans. Roy. Soc. London, vol. A247, pp. 529--551, April 1955.
  \bibitem{b2} J. Clerk Maxwell, A Treatise on Electricity and Magnetism, 3rd ed., vol. 2. Oxford: Clarendon, 1892, pp.68--73.
  \bibitem{b3} I. S. Jacobs and C. P. Bean, ``Fine particles, thin films and exchange anisotropy,'' in Magnetism, vol. III, G. T. Rado and H. Suhl, Eds. New York: Academic, 1963, pp. 271--350.
  \bibitem{b4} K. Elissa, ``Title of paper if known,'' unpublished.
  \bibitem{b5} R. Nicole, ``Title of paper with only first word capitalized,'' J. Name Stand. Abbrev., in press.
  \bibitem{b6} Y. Yorozu, M. Hirano, K. Oka, and Y. Tagawa, ``Electron spectroscopy studies on magneto-optical media and plastic substrate interface,'' IEEE Transl. J. Magn. Japan, vol. 2, pp. 740--741, August 1987 [Digests 9th Annual Conf. Magnetics Japan, p. 301, 1982].
  \bibitem{b7} M. Young, The Technical Writer's Handbook. Mill Valley, CA: University Science, 1989.
\end{thebibliography}
\vspace{12pt}
\color{red}
IEEE conference templates contain guidance text for composing and formatting conference papers. Please ensure that all template text is removed from your conference paper prior to submission to the conference. Failure to remove the template text from your paper may result in your paper not being published.

\end{document}
