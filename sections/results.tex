\documentclass[../main.tex]{subfiles}
\graphicspath{{\subfix{../images/}}}
\begin{document}

  As equações deverão ser escritas em itálico com numeração consecutiva entre parênteses, rente à margem direita. Equações com mais de uma linha deverão ser numeradas na última linha, entre parênteses e rente à margem direita.

  \begin{equation}
    PV = \eta RT
  \end{equation}

  Caso necessário, a lista de notações e símbolos utilizados, assim como  suas unidades de medida, deverá ser relacionada antes das referências  bibliográficas por ordem alfabética. Vide NBRs 15287 e 14724.


  As tabelas deverão ser justificadas (usando toda a área entre as  margens).  As  tabelas  poderão  ser  coloridas  ou  em  preto  e  branco.  O  título  da  tabela  deve  ser  alinhado  à  margem  esquerda  e  digitado  em  fonte  Times  New  Roman  /  Arial  11.  As  unidades  de  medida  correspondentes  a  todos  os  termos  deverão ser claramente indicadas, usando o sistema internacional (S.I.). 

  \begin{table}
    \caption{Titulo da Tabela}
    \centering
    \begin{tabular}{ |c|c|c|c| } 
      \hline
      col1 & col2 & col3 \\
      \hline
      \multirow{3}{4em}{Multiple row} & cell2 & cell3 \\ 
      & cell5 & cell6 \\ 
      & cell8 & cell9 \\ 
      \hline
      \end{tabular}
    
      Fonte: Proprio autor
    \label{tab:table-name}
  \end{table}

\end{document}