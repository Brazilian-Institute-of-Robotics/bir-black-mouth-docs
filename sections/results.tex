\documentclass[../main.tex]{subfiles}
\graphicspath{{\subfix{../images/}}}
\begin{document}

  A etapa de resultados e análises tem como objetivo realizar testes e coletar dados para a análise de performance dos algoritmos implementados no robô. Durante essa etapa, será analisada a performance do robô no mundo real de forma estatistica, por meio de três testes. O primeiro consiste em observar a capacidade do protótipo em mover o pé por uma dada trajetória. No segundo teste é analisado se o robô atende a uma velocidade solicitada, utilizando dois terrenos diferentes e observando a oscilação de estabilidade do corpo ao longo da trajetória. Por fim, foi analisado a perfomance do Caramelo caminhando por terrenos inclinados e sua capacidade de ultrapassar pequenos obstáculos.

  \subsection{Análise de trajetória}
  Durante este teste, o corpo do protótipo está estático e suspenso por um suporte e então é solicitado que o pé do robô percorra duas trajetórias curvilíneas calculadas pelo algoritimo proposto, até os pontos x1,y1 e x2,y2, detalhados na tabela X.

  APRESENTAR TABELA E CONSIDERAÇÕES

  \subsection{Controle de velocidade}

  Para o teste do controle de velocidade, foi enviado um comando de velocidade linear, na direção x, e medido o tempo que o robô levou para alcançar 1,5 m, dessa forma, obteve-se a velocidade média do robô. Neste experimento parte dos testes aconteceram utilizando o controle de estabilidade proposto e a outra parte sem, a fim de analisar a contribuição deste metódo no equilibrio do robô ao desempenhar uma trajetória. Os testes ocorreram com os mesmos paramêtros de passo do teste anterior, e em dois terrenos distintos, um terreno regular, feito de concreto e outro irregular, composto por pequenas pedras, como demonstra a imagem W.

  APRESENTAR TABELA E CONSIDERAÇÕES
  
  \subsection{Teste 3}

  % \begin{table}
  %   \caption{Titulo da Tabela}
  %   \centering
  %   \begin{tabular}{ |c|c|c|c| } 
  %     \hline
  %     col1 & col2 & col3 \\
  %     \hline
  %     \multirow{3}{4em}{Multiple row} & cell2 & cell3 \\ 
  %     & cell5 & cell6 \\ 
  %     & cell8 & cell9 \\ 
  %     \hline
  %     \end{tabular}
    
  %     Fonte: Proprio autor
  %   \label{tab:table-name}
  % \end{table}

\end{document}