\documentclass[../main.tex]{subfiles}
\graphicspath{{\subfix{../images/}}}
\begin{document}

  A seguir, serão apresentados os resultados preliminares e os experimentos realizados com o Caramelo. 
  
  \subsection{Testes preliminares}
  
  Os resultados preliminares referem-se aos resultados obtidos com as funcionalidades básicas do robô: capacidade de movimentação a partir do modelo cinemático e capacidade de controle da orientação do corpo com o controlador de angulação.

  Primeiramente, foi avaliada a movimentação do corpo a partir do modelo cinemático. Foi observado que o sistema não só é capaz de realizar a cinemática de cada uma das pernas individualmente como também do seu corpo em todos os 6 graus de liberdade (translações em $x$, $y$ e $z$ e rotações em $roll$, $pitch$ e $yaw$). A figura \ref{fig:moving_body} ilustra o movimento do corpo do robô em cada um desses graus de liberdade.

  \begin{figure}[!htb]
    \centering
    \caption{Movimentação do corpo do robô.}
    \includegraphics[width=0.48\textwidth]{moving_body.png}
    
    Fonte: autores.
    \label{fig:moving_body}
  \end{figure}

  A seguir, foi avaliada a performance do controlador de angulação do corpo. Os gráficos da figura \ref{fig:grafico_controlling} ilustram o comportamento do sistema à variação dos \textit{setpoints} de orientação em \textit{roll} e em \textit{pitch} para o corpo do robô ao longo do tempo. As curvas em azul representam os \textit{setpoints} aplicados como sinais degrau, enquanto que as curvas em vermelho ilustram o comportamento do sistema.
  
  Nota-se que o protótipo é capaz de se adaptar rapidamente aos novos valores desejados de orientação simultaneamente em ambos os eixos.

  \begin{figure}[!htb]
    \centering
    \caption{Respostas dos controles de angulação.}
    \begin{subfigure}[t]{0.48\textwidth}
      \centering
      \includegraphics[width=1.0\textwidth]{grafico_controlling_X.png}
      \caption{\textit{Roll}}
      \label{fig:controlling_roll}
    \end{subfigure}
    \begin{subfigure}[t]{0.48\textwidth}
      \centering
      \includegraphics[width=1.0\textwidth]{grafico_controlling_Y.png}
      \caption{\textit{Pitch}}
      \label{fig:controlling_pitch}
    \end{subfigure}
    
    Fonte: autores.
    \label{fig:grafico_controlling}
  \end{figure}

  \subsection{Experimentos}
  Em seguida, serão apresentados os resultados dos experimentos descritos na seção \ref{sec:method_results_analysis}.

  \subsubsection{Análise de trajetória}
  Durante este experimento, o objetivo foi analisar o tempo real de execução da trajetória, as coordenadas finais $(x_{final}, y_{final})$ da pata e a altura máxima $z_{max}$ atingida durante o passo, comparando-os com os valores esperados para cada um desses parâmetros. A tabela \ref{tab:trajetoria} mostra as médias encontradas para cada um desses parâmetros durante a realização do experimento.

  \begin{table}[!htb]
    \caption{Resultados do experimento da trajetória da pata.}
    \centering
    \begin{tabular}{ccc}
      \hline
      Teste & 1         & 2        \\
      \hline
      $\bar{tempo}$ (s)          & 0,557  & 0,557 \\
      \hline
      $\sigma_{tempo}$ (s)       & 0,008  & 0,008 \\
      \hline
      $\bar{x}_{final}$ (cm)     & 5,034  & 3,024 \\
      \hline
      $\sigma_{x_{final}}$ (cm)  & 0,009  & 0,013 \\
      \hline
      $\bar{y}_{final}$ (cm)     & 2,924  & 4,915 \\      
      \hline
      $\sigma_{y_{final}}$ (cm)  & 0,013  & 0,004 \\      
      \hline
      $\bar{z}_{max}$ (cm)       & 4,760  & 4,755 \\      
      \hline
      $\sigma_{z_{max}}$ (cm)    & 0,023  & 0,028 \\
      \hline   
    \end{tabular}

    Fonte: autores.
    \label{tab:trajetoria}
  \end{table}

  Inicialmente, foram removidos os \textit{outliers} das amostras, valores que fogem da normalidade e podem prejudicar a análise dos dados. Em seguida, a fim de avaliar a normalidade dos dados, foi empregado o teste de \textit{Shapiro-Wilk}, o qual indicou que as amostras de ambos os testes estão semelhantes a uma distribuição normal, considerando um nível de confiança de $95\%$.  Foram realizadas dois testes de análise de variância (ANOVA) unilaterais, o primeiro relacionando o tempo de execução da trajetória nos experimentos 1 e 2, e o segundo relacionando a altura máxima alcançada, com o intuito de verificar se os resultados se alteram para diferentes valores de $(x, y)$ requisitados. O resultado da ANOVA indica um $f_{valor}$ de aproximadamente $0,7212$, em se tratando de tempo, e de $0,5070$ para a altura máxima alcançada durante o passo, ambos valores superiores a $0,05$. Implica-se então que não há uma diferença significativa entre as médias em cada experimento, o que significa que diferentes comandos de trajetórias em $(x, y)$ não interferem nos resultados de tempo e na altura máxima do passo.

  O gráfico da figura \ref{fig:grafico_trajetoria_xyz} representa uma das amostras coletada para o primeiro caso ($x=0,05m$, $y=0,03m$), correspondente à trajetória em $z$ (altura do passo). Nota-se que, como esperado pela análise da tabela \ref{tab:trajetoria}, há um atraso na execução da trajetória (neste caso específico de aproximadamente $54ms$) e a altura máxima alcançada é levemente inferior à desejada (alcançando neste caso um valor próximo a $0,0476m$).

  \begin{figure}[!htb]
    \centering
    \caption{Trajetórias realizadas pelas patas.}
    % \includegraphics[width=0.48\textwidth]{grafico_trajetoria_xy.png}
    \includegraphics[width=0.48\textwidth]{grafico_trajetoria_z.png}
  
    Fonte: autores.
    \label{fig:grafico_trajetoria_xyz}
  \end{figure}

  \begin{figure}
    \centering
    \caption{Tempo e altura da pata durante um passo.}
    \includegraphics[width=0.5\textwidth]{tempo_zmax_traj.png}
    Fonte: autores.
    \label{fig:time_zmax_traj}
  \end{figure}

  \begin{figure}[!htb]
    \centering
    \caption{Oscilação do corpo em ambos os tipos de terreno.}
    \begin{subfigure}[t]{0.49\textwidth}
      \centering
      \includegraphics[width=1.0\textwidth]{plane_boxplot.png}
      \caption{Terreno plano}
      \label{fig:imu_test_plane}
    \end{subfigure}
    \begin{subfigure}[t]{0.49\textwidth}
      \centering
      \includegraphics[width=1.0\textwidth]{irregular_boxplot.png}
      \caption{Terreno irregular}
      \label{fig:imu_test_irregular}
    \end{subfigure}
    
    Fonte: autores.
    \label{fig:imu_test}
  \end{figure}

  \subsubsection{Controle de velocidade}
  Durante o experimento de controle de velocidade foi observada uma caminhada eficiente em ambos os terrenos, porém, mais linear e com menor trepidação em terrenos planos do que em terrenos irregulares. Por meio dos dados coletados, foram extraídas as informações da velocidade média da caminhada para cada amostra, oscilação máxima de rotação do corpo em \textit{roll} e \textit{pitch} e os respectivos desvios padrões, apresentados na Tabela \ref{tab:vel_stab}. Assim, foi observado que o robô se aproximou mais vezes da velocidade solicitada no terreno plano combinado ao controle de angulação, enquanto que no irregular, o teste com controle de angulação ativo apresentou uma velocidade média inferior e baixa consistência nos dados. A diferença percebida entre a velocidade solicitada e a inferida é justificada pelo acúmulo de erro nas coordenadas finais $(x_f, y_f)$ das patas a cada passo. Ou seja, quando levado em consideração o próprio peso do robô, o sistema de controle não é capaz de seguir fielmente a trajetória do passo.

  \begin{table}[!htb]
    \caption{Resultados do experimento do controlde de velocidade.}
    \centering
    \begin{tabular}{ccccc}
      \hline
      Teste & 1 & 2 & 3 & 4 \\ \hline
      Ter. & Reg. & Reg. & Irreg. & Irreg. \\ \hline
      \begin{tabular}[c]{@{}c@{}}C. R.\end{tabular} & Não & Sim & Não & Sim \\ \hline
      \begin{tabular}[c]{@{}c@{}}Vel. \\ (cm/s) \end{tabular} &   2,92   &  3,33  &   2,97   & 2,77  \\ \hline
      \begin{tabular}[c]{@{}c@{}} $\sigma_{Vel}$  \\ (cm/s) \end{tabular} & 0,135 & 0,079 & 0,197 & 0,241 \\ \hline
      \begin{tabular}[c]{@{}c@{}} $\Delta_{Roll}$ \end{tabular} & 12,54\degree & 8,74\degree & 13,90\degree & 12,17\degree \\\hline
      \begin{tabular}[c]{@{}c@{}} $\sigma_{Roll}$ \end{tabular}  & 2,42\degree & 2,14\degree & 1,80\degree & 3,27\degree \\ \hline
      \begin{tabular}[c]{@{}c@{}} $\Delta_{Pitch}$ \end{tabular} & 11,46\degree & 8,29\degree & 15,19\degree & 13,18\degree \\ \hline
      \begin{tabular}[c]{@{}c@{}} $\sigma_{Pitch}$ \end{tabular}  & 1,83\degree & 2,39\degree & 1,44\degree & 2,94\degree \\ \hline

    \end{tabular}
    Fonte: autores.
    \label{tab:vel_stab}
  \end{table}

  Ainda na Tabela \ref{tab:vel_stab}, foi observado que a menor oscilação em \textit{roll} e em \textit{pitch} ocorreu no terreno plano utilizando o controle de angulação, enquanto que a maior oscilação ocorreu nas condições opostas (teste 3). Seguindo com a análise dos dados, assim como no primeiro experimento, foram removidos os \textit{outliers} das amostras e aplicado o teste de normalidade de \textit{Shapiro-Wilk}. A fim de avaliar se o controle de rotação implementado contribuiu na estabilidade da caminhada, foi utilizada a ANOVA para comparar os testes 1 e 2 (mostrados na tabela), e os testes 3 e 4. O resultado desta análise aponta que a hipótese nula de que o controle de rotação não influencia na estabilidade do robô pode ser rejeitada, uma vez que o $f_{valor}$ para ambas as análises se mostrou menor que $0,05$, (incluir valores). Desse modo, comprova-se que o controle de angulação influencia positivamente na estabilidade do robô

    
  % \subsection{Testes complementares}
  % Como resultados complementares, foi observado que o robô possui a habilidade de andar por terrenos inclinados. Durante o teste, o robô foi capaz de se locomover por um plano inclinado com aproximadamente 5,3$\degree$ de inclinação (figura \ref{fig:tests3-4}). 

  % O último teste constatou que o robô é capaz de ultrapassar pequenos obstáculos. Para isso, foram considerados degraus de três diferentes alturas: 2, 4 e 5 $cm$ (figura \ref{fig:tests3-4}). O robô foi capaz de superar degraus de 2 e 4 $cm$, porém falhou ao tentar subir um de 5 $cm$.

  % \begin{figure}[!htb]
  %   \centering
  %   \caption{Fotos do robô no teste da rampa e do degrau.}
  %   \includegraphics[width=0.48\textwidth]{ramp_test.jpeg}
  %   \includegraphics[width=0.48\textwidth]{test4.png}

  %   Fonte: autores
  %   \label{fig:tests3-4}
  % \end{figure}

\end{document}