\documentclass[../main.tex]{subfiles}
\graphicspath{{\subfix{../images/}}}
\begin{document}
Com o avanço da robótica, robôs móveis estão cada vez mais tomando espaço em setores chave da economia como o comercial, industrial e militar. Os robôs terrestres que se locomovem com pernas têm se mostrado mais eficientes para se locomover em terrenos irregulares, inclinados e escorregadios e também para superar obstáculos \cite{X.134}. Quando comparado com robôs com rodas, robôs com pernas ainda possuem melhor mobilidade e manobrabilidade em ambientes complexos, possibilitando transitar por caminhos não necessariamente contínuos. Entre os robôs com pernas, os quadrúpedes vêm ganhando destaque por apresentarem maior estabilidade e uma estrutura mais simples que os bípedes e hexápodes \cite{Shi2021}.

O BigDog, desenvolvido pela Boston Dynamics em 2004, foi um dos primeiros robôs quadrúpedes a navegar em terrenos irregulares, incluindo trilhas e solos com neve ou lama. Desde então, as atuais soluções disponíveis no mercado apresentam maior maturidade de tecnologia. O Spot, por exemplo, é um robô quadrúpede para missões \textit{indoor} e \textit{outdoor} fabricado pela empresa \textit{Boston Dynamics}. Ele é capaz de alcançar uma velocidade de até $1,6 m/s$ e se equilibrar em ambientes incertos carregando até $14 kg$ \cite{Spot}.  Semelhante ao Spot, o Anymal, desenvolvido pela \textit{Anybotics}, consegue atuar de forma autônoma em ambientes difíceis. Ele possui $70 cm$ de altura e apresenta uma capacidade de \textit{payload} que comporta simples sensores até manipuladores \cite{Fankhauser2018}.

O uso de uma plataforma com quatro pernas requer um sistema de locomoção robusto que envolve o controle de equilíbrio da plataforma, o controle das juntas do robô e o planejamento de marchas. Para poder aplicar todos esses conceitos e desenvolver aplicações reais com robôs quadrúpedes, é fundamental estudar sua estrutura física, sua cinemática, os tipos de marcha que eles podem realizar e os métodos de controle de locomoção que são utilizados nesses robôs.

O objetivo deste trabalho é desenvolver um sistema robótico do tipo quadrúpede capaz de se locomover de forma estável quando teleoperado. Serão estudados aspectos construtivos, de locomoção e de controle desse tipo robô. Além disso, será projetado um protótipo que deve ser simulado e fabricado, a fim de testar os algoritmos de locomoção não apenas em um ambient virtual, mas também na prática. Por fim, serão realizados testes de performance de locomoção com o protótipo físico, os quais analisarão a estabilidade do robô durante sua locomoção em ambientes de terreno plano e irregular.

Com o objetivo de guiar o desenvolvimento do projeto, foram elencados alguns requisitos que o robô deve cumprir:

\setlist{nolistsep}
\begin{itemize}[noitemsep]
  \item Ser capaz de operar por tempo suficiente para inspecionar um ambiente de $50m^2$;
  \item Ser capaz de se locomover em ambientes irregulares;
  \item Ser capaz de transpor obstáculos pequenos (máx. $5cm$);
  \item Ser capaz de operar em ambientes \textit{indoor} e \textit{outdoor};
  \item Ser capaz de ser transportado por apenas uma pessoa;
  \item Ter capacidade de \textit{payload} de $2kg$.
\end{itemize}

\end{document}
