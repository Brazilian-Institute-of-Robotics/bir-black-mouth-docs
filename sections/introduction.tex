\documentclass[../main.tex]{subfiles}
\graphicspath{{\subfix{../images/}}}
\begin{document}

  Com o avanço da robótica, robôs móveis estão, cada vez mais, tomando espaço em setores chave da economia como o comercial, industrial e militar. Os robôs terrestres que se locomovem com pernas tem se mostrado mais eficientes quanto a locomoção em terrenos irregulares, inclinados e escorregadios e quanto à superação de obstáculos \cite{X.134}. Quando comparado com robôs com rodas, robôs quadrupedes ainda possuem melhor mobilidade e manobrabilidade em ambientes complexos, possibilitando transitar por caminhos não necessariamente contínuos.
          
  O BigDog, desenvolvido pela Boston Dynamics em 2004, foi um dos primeiros robôs quadrupedes  robustos a navegar em superfícies diversas, incluindo trilhas e solos com neve ou lama. Desde então, as atuais soluções disponíveis no mercado apresentam alta maturidade, como por exemplo o Spot, um robô quadrupede para missões indoor e outdoors, fabricado pela Boston Dynamics. Ele é capaz de alcançar uma velocidade de até 1.6 m/s e se equilibrar dinamicamente em ambientes incertos carregando até 14 kg \textbf{\textcolor{red}{Fonte: bostondynamics.com/products/spot}}.  Na mesma linha de robôs cachorros de médio porte, encontra-se o Anymal, desenvolvido pela AnyRobotics para atuar de forma autônoma em ambientes difíceis. Este equipamento possui 70 cm de altura apresenta um range de payload que comporta simples sensores até complexos braços robóticos \textbf{\textcolor{red}{Fonte: ANYmal A unique quadruped robot conquering harsh environments}}

  Entre o robôs com pernas, os robôs quadrúpedes vêm ganhando destaque por apresentarem maior estabilidade e uma estrutura mais simples que os bípedes e hexapodes \cite{Shi2021}. Além disso, conseguem realizar vários tipos de passadas, as quais podem lhe fornecer mais estabilidade, velocidade ou eficiência energética. 

  O uso de uma plataforma com quatro pernas requer um controle de locomoção robusto que envolve o controle de equilíbrio da plataforma, o controle das juntas do robô e o planejamento de marchas. A locomoção de robô quadrúpedes é um tema bastante amplo e que pode ser abordado de várias perspectivas, sendo essas as mais comuns e presentes em várias aplicações desse tipo de sistema robótico. Para poder aplicar todos esses conceitos e desenvolver aplicações reais com robôs quadrúpedes, é fundamental estudar a estrutura física, a cinemática, os tipos de passo e os métodos de controle de movimentação desses robôs. 

  O objetivo deste projeto é desenvolver um sistema robótico do tipo quadrúpede capaz de realizar uma caminhada de forma teleoperada e estável, o que envolve o estudo da cinemática, design mecânico e controle de locomoção deste tipo de robô. Para isso, o modelo do protótipo é simulado de forma a embasar a construção de um protótipo físico. O protótipo físico é de pequeno porte, possui 3 DOFs (graus de liberdade) por perna e é atuado servomotores. Ele possui a habilidade de se locomover de maneira estável em marcha XXX por ambientes planos e irregulares \textit{indoor} e \textit{outdoor}. Além disso, o robô é capaz de superar pequenos obstáculos \textcolor{red}{e transportar cargas de até X kg.}

  \textcolor{red}{\textbf{--}
  \begin{itemize}
    \item Requisitos devem estar aqui?
  \end{itemize}
  }

\end{document}
