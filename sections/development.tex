\documentclass[../main.tex]{subfiles}
\graphicspath{{\subfix{../images/}}}
\begin{document}

Baseado nos conceitos apresentados, foi projetado, simulado e desenvolvido um robô quadrúpede nomeado de Caramelo (figuras \ref{fig:terrenos}, \ref{fig:caramel_body} e \ref{fig:moving_body}). Este plataforma é um robô quadrúpede de pequeno porte voltado para pesquisa e educação. 
Durante o processo, a arquitetura de hardware do robô foi desenhada, assim como sua estrutura física, a qual foi produzida em plástico ABS por uma impressora 3D. Os atuadores do robô são servomotores do modelo \textit{dynamixel} MX-28 e sua central de processamento é composta por uma \textit{RaspberryPi} 4. Além disso, a plataforma conta com um sensor inercial modelo MPU6050. Todo o \textit{software} foi desenvolvido com o \textit{Robot Operating System 2 Humble} (\textit{ROS2}) \cite{ROS2Humble}, um consolidado \textit{framework} de robótica.

A estrutura do robô é do tipo mamífero, isto quer dizer que ele possui duas juntas no plano sagital, de forma similar a animais como cachorros e cavalos. As pernas do robô obedecem uma configuração chamada de \textit{full-elbow}, na qual todas as quatro pernas são orientadas para trás, assim como os cotovelos. Além disso, ele possui 3 GDL por perna, o que permite uma grande liberdade de movimentação para as patas. Seu \textit{design} foi pensado para favorecer o balanço de massas entre o corpo e as pernas do robô, ou seja, a maior parte da massa se encontra no corpo ou próxima a ele. Pernas mais leves podem se movimentar rapidamente sem alterar, de forma significativa, o centro de gravidade do robô, o que aumenta a estabilidade e requer menos complexidade de controle. Em contra partida, elas devem ser também resistentes o suficiente para suportar o peso do robô e o impacto das patas no chão. Os componentes eletrônicos internos, que abrangem sensores, unidades de processamento e a interface de comunicação com os atuadores, foram dispostos de forma simétrica, a fim de manter o centro de massa o mais próximo do centro do corpo. Os motores (componentes que contribuem com a maior massa para o sistema) foram dispostos o mais próximo possível do corpo. Um destaque para o motor que atua na junta da tíbia foi sua instalação na parte superior do fêmur, com o objetivo de diminuir o momento de inércia da perna. Essa escolha demandou a adição de um sistema de transmissão entre o eixo do motor e a tíbia, formado por uma haste rígida de metal com uma junta esférica em cada extremidade.

A marcha desenvolvida para o robô segue uma sequência de etapas demonstrada na figura \ref{fig:trot_pattern}, cujas áreas em branco representam a etapa de \textit{swing} e as em cinza a de \textit{stance}. Durante a fase stance, as pernas estão no solo e impulsionam o robô para frente. Na etapa de swing, as pernas são erguidas para deslocar a pata até o próximo ponto de apoio. Esta é a marcha principal do robô, chamada de trote, embasada no conceito de marchas periódicas e simétricas. As marchas periódicas são caracterizadas pela repetição contínua dos mesmos movimentos nos mesmos instantes. Já a simetria é uma característica de marchas que movimentam um par de pernas em conjunto, saindo e voltando para o solo de forma sincronizada. Embora a estrutura do robô permita a realização de  outros tipos de marcha, neste trabalho, foi considerada apenas o \textit{trot}, devido a sua simplicidade e eficiência. Ainda assim, com o objetivo de diminuir a complexidade do controle de locomoção, foi adotada uma marcha descontínua, ou seja, o corpo do robô se desloca apenas quando todas as patas estão no solo realizando um movimento intermitente.  É possível perceber ainda na figura \ref{fig:trot_pattern} que sempre o mesmo par de pernas diagonais se move no mesmo instante. Entre duas etapas consecutivas de \textit{swing}, há um momento em que todas as patas estão em \textit{stance}, que é quanto o corpo do robô é deslocado no sentido desejado de locomoção.

\begin{figure}[!htb]
  \centering
  \caption{Padrão de movimentação da marcha para cada perna.}
  \includegraphics[width=0.45\textwidth]{trot_pattern.png}
  \vfill
  Fonte: autores.
  % \vspace{-\baselineskip}
  \label{fig:trot_pattern}
\end{figure}

O subsistema de controle do robô é composto por três elementos principais: o planejador de trajetória, o modelo cinemático e o controlador de angulação do corpo do robô. O controle de angulação do corpo do robô é responsável por controlar o ângulo de \textit{pitch} (rotação em $y$) e o de \textit{roll} (rotação em $x$), de modo a manter o robô em $0^{\circ}$ a todo momento. O planejador de trajetória é o responsável por controlar cada pata do robô e, por consequência, o corpo.  Ele calcula a trajetória que cada pata deve realizar, com base nas etapas de \textit{stance} e \textit{swing}, de forma a mover o robô na direção e velocidade desejada. (Além disso, o planejador de trajetória também considera o esforço de controle enviado pelos controladores de angulação). Por fim, o modelo cinemático do robô é responsável por mapear a posição tridimensional de cada pata e obter a angulação das juntas que a compõe. 

(O planejador envia os pontos em que cada pata devem estar a uma frequência de $\SI{50}{\hertz}$. Essa frequência foi definida por ser a maior que o sistema de processamento do robô foi capaz de suportar e por respeitar o teorema Nyquist, que indica que, nesse caso, a frequência deve ser maior do que $\SI{4}{\hertz}$ --- já que o período do passo é $0.5 s$. )

A seguir, será apresentado o desenvolvimento do modelo cinemático, dos controladores de angulação e da trajetória que cada pata realiza na etapa de \textit{swing}.

\subsection{Modelo cinemático do Caramelo}
\label{sec:detail_inv_kinematics}

Como dito anteriormente, o modelo cinemático é utilizado para resolver a cinemática inversa e a cinemática direta do robô. Para a cinemática direta, foi utilizado o pacote \textit{tf2}, um recurso disponível no \textit{ROS2} que facilita o gerenciamento de transformações entre eixos de coordenadas. A cinemática inversa, por outro lado, foi feita com base em uma análise geométrica. As variáveis $\theta_1$, $\theta_2$ e $\theta_3$ expressam a posição angular de cada uma das juntas de uma perna do robô e são calculadas com auxílio das equações \ref{eq:theta1} a \ref{eq:B} em função da posição $(x_{IK}, y_{IK}, z_{IK})$ desejada para a pata e dos comprimentos $L_1$, $L_2$ e $L_3$ (figura \ref{fig:caramel_tfs}).
\begin{equation}
  \label{eq:theta1}
  \theta_1 = \arctan{(\frac{x_{IK}}{y_{IK}})} - \arctan{(\frac{L_1}{a})}
\end{equation}
\begin{equation}
  \label{eq:theta2}
  \theta_2 = \frac{\pi}{2} - \arctan{(\frac{a}{z_{IK}}}) - \arctan{(\frac{\sqrt{1-A^2}}{A})}
\end{equation}
\begin{equation}
  \label{eq:theta3}
  \theta_3 = \arctan(\frac{\sqrt{1-B^2}}{B})
\end{equation}
\begin{equation}
  \label{eq:a}
  a = \sqrt{x_{IK}^2+y_{IK}^2-L_1^2}
\end{equation}
\begin{equation}
  \label{eq:A}
  A =\frac{a^2+z^2+L_2^2-L_3^2}{2L_2\sqrt{a^2+z_{IK}^2}}
\end{equation}
\begin{equation}
  \label{eq:B}
  B = \frac{a^2+z_{IK}^2-L_2^2-L_3^2}{2L_2L_3}
\end{equation}

\begin{figure}[!htb]
  \centering
  \caption{Links da perna do robô.}
  \includegraphics[width=0.4\textwidth]{caramel_tfs.png}
  
  Fonte: autores.
  \label{fig:caramel_tfs}
\end{figure}

Essas equações são úteis para o cálculo da posição de uma única perna, mas são insuficientes para realizar a cinemática do corpo do robô. Desta forma, um \textit{frame} central, chamado de \textit{base\_link} (figura \ref{fig:caramel_body}), é utilizado como referência, e uma matriz $T_M$ (eqs. \ref{eq:Tm} e \ref{eq:Rxyz}) é utilizada para realizar a cinemática do corpo, a partir das translações $(x_c, y_c, z_c)$ e rotações $(\alpha, \beta, \gamma)$ desejadas, sendo possível controlar cada um dos 6 graus de liberdade. Para tanto, as transformações $T_{FR}$, $T_{FL}$, $T_{BL}$ e $T_{BR}$ de cada um dos ombros (\textit{hip\_links}) em relação ao \textit{base\_link} são necessárias.
\begin{equation}
  \label{eq:Tm}
  T_M =
  \begin{bmatrix}
      &         &   & x_c \\
      & R_{xyz} &   & y_c \\
      &         &   & z_c \\
    0 & 0       & 0 & 1
  \end{bmatrix}
\end{equation}
\begin{equation}
  \label{eq:Rxyz}
  \begin{split}
    R_{xyz} =
    \begin{bmatrix}
      1 & 0          & 0           \\
      0 & \cos\alpha & -\sin\alpha \\
      0 & \sin\alpha & \cos\alpha
    \end{bmatrix}
    \\.
    \begin{bmatrix}
      \cos\beta  & 0 & \sin\beta \\
      0          & 1 & 0         \\
      -\sin\beta & 0 & \cos\beta
    \end{bmatrix}
    \\.
    \begin{bmatrix}
      \cos\gamma & -\sin\gamma & 0 \\
      \sin\gamma & \cos\gamma  & 0 \\
      0          & 0           & 1
    \end{bmatrix}
  \end{split}
\end{equation}

\begin{figure}[!htb]
  \centering
  \caption{Eixos do robô em posição de repouso.}
  \vspace{-0.75cm}
  \includegraphics[width=0.35\textwidth]{caramel_body.drawio.png}
  
  Fonte: autores.
  \label{fig:caramel_body}
\end{figure}

O cálculo das angulações de cada perna então é feito utilizando como entrada os valores $(x_{IK}, y_{IK}, z_{IK})$ resultantes de cada uma das transformações, conforme a equação \ref{eq:xyzik}. O mesmo cálculo é feito para as demais pernas, utilizando  $T_{FL}$, $T_{BL}$ e $T_{BR}$.
\begin{equation}
  \label{eq:xyzik}
  \begin{bmatrix}
    x_{IK} \\
    y_{IK} \\
    z_{IK} \\
    1
  \end{bmatrix}= (T_M.T_{FR})^{-1}.
  \begin{bmatrix}
    x \\
    y \\
    z \\
    1
  \end{bmatrix}
\end{equation}

Desta forma, a cinemática inversa é capaz de computar as angulações  $\theta_1$, $\theta_2$ e $\theta_3$ de cada uma das pernas a partir da posição $(x, y, z)$ das patas em relação ao link central do robô e às translações $(x_c, y_c, z_c)$ e rotações $(\alpha, \beta, \gamma)$ desejadas para o corpo. Entretanto, em muitos casos, é mais conveniente realizar o cálculo dos ângulos passando como entrada as posições $(x, y, z)$ das patas em relação à sua posição \textit{default}, ou seja, a posição do seu \textit{foot\_link} quando o robô está em seu estado de repouso (figura \ref{fig:caramel_body}). Para isso, é possível realizar, previamente ao cálculo das angulações, mais uma transformação, desta vez do \textit{base\_link} para cada uma das posições \textit{default} das patas (equação \ref{eq:xyzik_foot}).
\begin{equation}
  \label{eq:xyzik_foot}
  \begin{bmatrix}
    x_{ik} \\
    y_{ik} \\
    z_{ik} \\
    1
  \end{bmatrix}= (T_M.T_{FR})^{-1}.
  (F_{FR}.
  \begin{bmatrix}
    x \\
    y \\
    z \\
    1
  \end{bmatrix})
\end{equation}

\subsection{Controle de angulação}

Os controladores de angulação são dois controladores PID em paralelo, responsáveis por controlar a rotação de \textit{roll} e \textit{pitch} do corpo do robô. Eles atuam de forma independente, controlando a rotação do corpo em ambos os eixos simultaneamente. Ambos os controladores são iguais e foram implementados seguindo o modelo apresentado no diagrama de blocos da figura \ref{fig:pid}.
\begin{figure}[!htb]
  \centering
  \caption{Controlador PID projetado.}
  \includegraphics[width=0.48\textwidth]{PID.drawio.png}
  Fonte: autores.
  \label{fig:pid}
  \vspace{-\baselineskip}
\end{figure}

O IMU é o sensor responsável por medir a rotação do corpo do robô, possibilitando a realimentação das saídas do sistema. O limitador foi adicionado para evitar que sejam enviados valores que extrapolam os limites de rotação das juntas do robô. Os esforços de controle são enviados para o planejador de marchas que, por sua vez, envia os comandos de movimentação para os controladores das juntas.

\subsection{Planejador de trajetória}

O planejador de trajetória é responsável por calcular a trajetória que cada pata deve realizar, tanto na fase de \textit{stance} quanto na fase de \textit{swing}. Para o Caramelo, a trajetória é uma curva cicloidal em ambas as etapas. Como apresentado em \cite{Shi2021}, uma curva cicloidal pode ser definida no espaço tridimensional entre os pontos $(x_o, y_o, z_o)$ e $(x_f, y_f, z_f)$ em função do tempo $t$ pelas equações (\ref{eq:traj_x}) a (\ref{eq:traj_k}), sendo $H$ a altura do passo e $T$ o período.
\begin{equation}
  x = (x_f - x_o) \frac{K - \sin{(K)}}{2 \pi} + x_o
  \label{eq:traj_x}
\end{equation}
\begin{equation}
  y = (y_f - y_o) \frac{K - \sin{(K)}}{2 \pi} + y_o
  \label{eq:traj_y}
\end{equation}
\begin{equation}
  z = H \frac{1 - \cos{(K)}}{2} + z_o
  \label{eq:traj_z}
\end{equation}
\begin{equation}
  K = \frac{2 \pi t}{T}
  \label{eq:traj_k}
\end{equation}

O gráfico de uma curva cicloidal no espaço 3D pode ser vista na figura \ref{fig:traj_space}. A mesma trajetória é ilustrada na figura \ref{fig:traj_time} em função do tempo. É possível perceber que a curva possui a primeira derivada nula no momento em que a pata toca o solo, o que é favorável ao controle de malha aberta, uma vez que quanto mais suave a aterrissagem, menos distúrbios são causados no sistema.
\begin{figure*}[h]
  \centering
  \caption{Trajetórias cicloidais para o passo de robô.}
  \begin{subfigure}[t]{0.32\textwidth}
    \centering
    \includegraphics[width=1.0\textwidth]{Cycloid_space.png}
    \caption{Curva no espaço 3D.}
    \label{fig:traj_space}
  \end{subfigure}
  \begin{subfigure}[t]{0.32\textwidth}
    \centering
    \includegraphics[width=1.0\textwidth]{Cycloid_time.png}
    \caption{Curva no tempo.}
    \label{fig:traj_time}
  \end{subfigure}
  \begin{subfigure}[t]{0.32\textwidth}
    \centering
    \includegraphics[width=1.0\textwidth]{Cycloid_modified.png}
    \caption{Curva com disposição de pontos modificada.}
    \label{fig:traj_time_modified}
  \end{subfigure}
  \vfill
  Fonte: autores.
  \vspace{-\baselineskip}
  \label{fig:traj_curve}
\end{figure*}

Além da altura, distâncias em $x$ e $y$ e período, o planejador de trajetória do Caramelo também conta com dois parâmetros que têm como objetivo melhorar ainda mais o controle da força com que a pata toca o chão. Como os atuadores do robô são servomotores controlados por posição, o torque é proporcional ao deslocamento que este deve realizar entre os pontos da trajetória. Ou seja, quanto maior a resolução da trajetória, mais suave será o movimento. No entanto, a resolução da trajetória $N$ é fixa, dada em função do período do passo e da frequência de controle do planejador de marchas ($\SI{50}{\hertz}$) $N = 50T$. Logo, a estratégia adotada é a de espaçar a mesma quantidade de pontos de forma desigual ao longo do período do passo, de forma que haja mais pontos próximos ao momento em que a pata aterrissa no solo, e menos pontos próximos ao momento em que ela é erguida. O parâmetro $P_T$ é uma fração do período total do passo, e o parâmetro $P_N$, a fração do número total de pontos do passo que deve se encontrar entre o tempo $0$ e $P_T \cdot T$. Em outras palavras, se $P_T = 0,66$ e $P_N = 0,33$, $33\%$ de $N$ estará nos primeiros dois terços do período, enquanto os $67\%$ restantes estarão no um terço final. A trajetória, considerando esses parâmetros, está ilustrada na figura \ref{fig:traj_time_modified}.

\end{document}
