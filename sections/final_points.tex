\documentclass[../main.tex]{subfiles}
\graphicspath{{\subfix{../images/}}}
\begin{document}

Para trabalhos futuros, recomenda-se a revisão dos atuadores usados no robô. Ao longo da fase de testes, foi observado que estes frequentemente falhavam por conta de sobrecarga, o que é um sinal de que foram subdimensionados. Por consequência, a capacidade de \textit{payload} não é significativa. Uma possível solução para essa questão é a substituição do modelo atual por um com maior torque e/ou a investigação da configuração das pernas do robô. A equipe observou que, como os motores que mais falhavam eram os traseiros, adotar a configuração \textit{elbow-knee} pode diminuir a carga neles, já que deixaria as pernas traseira em uma configuração simétrica às frontais. Espera-se, também, que isso influencie na performance de locomoção, deixando o robô mais rápido e estável. Além disso, sugere-se a aplicação de outros métodos de controle, como controladores baseados em modelos dinâmicos ou até em \textit{machine learning}.

\end{document}