\documentclass[../main.tex]{subfiles}
\graphicspath{{\subfix{../images/}}}
\begin{document}

Para trabalhos futuros, recomenda-se revisar os atuadores usados no robô. Ao longo da fase de testes, foi observado que eles frequentemente falhavam por conta de sobrecarga, o que é um sinal que eles foram subdimensionados. Por consequência, o a capacidade de \textit{payload} não é significativa. Uma possível solução para essa questão é substituir o modelo atual por um com maior torque e/ou investigar a configuração das pernas do robô. A equipe observou que, como os motores que mais falhavam eram os traseiros, adotar a configuração \textit{elbow-knee} pode diminuir a carga neles, já que deixaria as pernas traseira em uma configuração simétrica às frontais. Espera-se, também, que isso influencie na performance de locomoção, deixando o robô mais rápido e, talvez, com a capacidade de superar obstáculos maiores do que $4cm$.  

tecnicas avançadas de controle
controle de marcha

\end{document}