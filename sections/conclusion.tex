\documentclass[../main.tex]{subfiles}
\graphicspath{{\subfix{../images/}}}
\begin{document}

  Este trabalho abordou os conceitos utilizados para locomoção de robôs quadrúpedes e buscou aplicá-los no desenvolvimento de um robô real. O Caramelo foi desenvolvido para fins de educação e pesquisa na área de robôs com pernas, mais especificamente quadrúpedes. Trata-se de um projeto \textit{open source}, cujo código fonte está disponível publicamente no \textit{GitHub} \cite{caramel_repo}. 
  
  Os testes e experimentos realizados buscaram avaliar a performance da locomoção do robô no espaço tridimensional. Os testes preliminares indicaram que o robô possui a capacidade de mover o corpo em 6 GDL e controlar sua orientação em $roll$ e $pitch$. 
  
  Com base no primeiro experimento, pôde-se concluir que a pata é capaz de seguir uma trajetória até um ponto requisitado em $x$ e $y$ no cenário sem carga. Além disso, entre os dois testes realizados, não foram constatadas diferenças significativas no tempo total e na altura máxima da trajetória. Contudo, foi observado um atraso no tempo de execução trajetória, provavelmente relacionado ao atraso na resposta dos motores.
   
  O segundo experimento mostrou que o robô não alcançou a velocidade desejada em nenhum dos testes, estando o teste com terreno plano e controle de angulação ativo o mais próximo deste valor. Essa diferença pode estar relacionada com o fato de que o sistema de controle não conseguiu operar de forma adequada ao precisar sustentar o peso do robô. Como o controle da trajetória é feito em malha aberta (apenas os controladores individuais das juntas possuem malha fechada), não há compensação e a pata não alcança a distância esperada. Ademais, também pôde-se observar que o controle de angulação resultou numa maior estabilidade do robô ao operar em ambos os terrenos.
  
  Por meio dos resultados complementares, também é possível concluir que o Caramelo é capaz de superar planos levemente inclinados e degraus de até $4cm$ de altura. 

  Para trabalhos futuros, recomenda-se revisar os atuadores usados no robô. Ao longo da fase de testes, foi observado que eles frequentemente falhavam por conta de sobrecarga, o que é um sinal que eles foram subdimensionados. Por consequência, o a capacidade de \textit{payload} não é significativa. Uma possível solução para essa questão é substituir o modelo atual por um com maior torque e/ou investigar a configuração das pernas do robô. A equipe observou que, como os motores que mais falhavam eram os traseiros, adotar a configuração \textit{elbow-knee} pode diminuir a carga neles, já que deixaria as pernas traseira em uma configuração simétrica às frontais. Espera-se, também, que isso influencie na performance de locomoção, deixando o robô mais rápido e, talvez, com a capacidade de superar obstáculos maiores do que $4cm$.  

\end{document}