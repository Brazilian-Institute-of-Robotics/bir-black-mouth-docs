\documentclass[../main.tex]{subfiles}
\graphicspath{{\subfix{../images/}}}
\begin{document}

  Este trabalho abordou os conceitos utilizados para locomoção de robôs quadrúpedes e buscou aplicá-los no desenvolvimento de um robô real. O Caramelo foi desenvolvido para fins de educação e pesquisa na área de robôs com pernas, mais especificamente quadrúpedes. Trata-se de um projeto \textit{open source}, cujo código fonte está disponível publicamente no \textit{GitHub}. 
  
  Seu requisito principal era conseguir se locomover no espaço tridimensional de forma teleoperada e, a partir dos resultados apresentados, pode-se concluir que ele atendeu esse requisito. Os testes realizados com o protótipo buscaram avaliar a performance da locomoção do robô. Com base do primeiro teste, pôde-se concluir que o modelo cinemático é capaz calcular as posições das juntas necessárias para que a pata siga a trajetória e que o sistema de controle é capaz de conduzi-la através da trajetória, independente da distância requisitada em $x$ e $y$, no cenário sem carga. 
  
  O teste do controle de velocidade mostrou que o robô não alcançou a velocidade de $0,05 m/s$ em nenhum dos cenários, estando o cenário com chão plano e controle de rotação ativo o mais próximo deste valor. Essa diferença pode estar relacionada com o fato de que o sistema de controle não conseguiu operar de forma adequada ao precisar sustentar o peso do robô. Como o controle da trajetória é feito em malha aberta (apenas os controladores individuais das juntas possuem malha fechada), não há compensação e o passo acaba sendo menor do que o esperado. Ademais, também pôde-se observar que o controle de rotação resultou numa maior estabilidade do robô quando operando em terrenos planos, porém o mesmo não vale para o cenário com terreno irregular. Neste caso, o uso do controle de rotação não ocasionou em uma melhora tão significativa na estabilidade.
  
  Também é possível concluir que o Caramelo é capaz de superar planos inclinados e degraus de até 4 cm de altura. 

  Por fim, pode-se dizer que o robô cumpriu quatro dos seis requisitos definidos anteriormente. Embora ele não tenha conseguido superar obstáculos com $5cm$ de altura, ele conseguiu com 4 cm, o que demonstra um resultado promissor. Quanto a sua capacidade de \textit{payload}, a partir dos testes realizados e do que pôde ser observado pela equipe ao operar o robô, ele não é capaz de carregar uma carga de 2 kg. A massa total do robô é de aproximadamente 2 kg. Portanto, seriam necessários atuadores muito mais robustos para que ele pudesse sustentar $100\%$ do seu peso. Logo, conclui-se que esse requisito foi superdimensionado, considerando a escala do projeto proposto nesse trabalho.

  Para trabalhos futuros, recomenda-se revisar os atuadores usados no robô. Ao longo da fase de testes, foi observado que eles frequentemente falhavam por conta de sobrecarga, o que é um sinal que eles foram subdimensionados. Para corrigir essa questão, é possível substituir o modelo atual por um com maior torque e/ou investigar a configuração das pernas do robô. A equipe observou que, como os motores que mais falhavam eram os traseiros, adotar a configuração \textit{elbow-knee} pode diminuir a carga neles, já que deixaria as pernas traseira em uma configuração simétrica às frontais. Espera-se, também, que isso influencie na performance de locomoção, deixando o robô mais rápido e, talvez, com a capacidade de superar obstáculos maiores do que 4 cm.  

\end{document}