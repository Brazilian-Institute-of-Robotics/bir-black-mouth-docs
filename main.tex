\documentclass[a4paper, 12pt, twocolumn]{article}
\usepackage[top=2.5cm, bottom=3cm, left=2.5cm, right=2.5cm]{geometry}
\usepackage[utf8]{inputenc}
\usepackage[]{indentfirst}
\usepackage{graphicx}
\usepackage{subfiles}
\usepackage[brazilian]{babel}
\usepackage{times}
\usepackage{sectsty}
\usepackage{multirow}
\usepackage{xcolor}
\usepackage{amsmath}
\usepackage[abnt-emphasize=bf, num]{abntex2cite}
\usepackage{subcaption}
\usepackage{gensymb}
\usepackage{titlesec}
\usepackage{enumitem}

\graphicspath{ {./images/} }

% \pagenumbering{gobble}
\setlength{\parskip}{0.2cm}
\setlength{\parindent}{1.3cm}
\sectionfont{\fontsize{12}{15}\selectfont}
\subsectionfont{\fontsize{12}{15}\selectfont}
\subsubsectionfont{\fontsize{12}{15}\selectfont}
\titlespacing*{\section}
{0pt}{0.3\baselineskip}{0.3\baselineskip}
\titlespacing*{\subsection}
{0pt}{0.3\baselineskip}{0.3\baselineskip}
\titlespacing*{\subsubsection}
{0pt}{0.3\baselineskip}{0.3\baselineskip}

\captionsetup{belowskip=-0.0pt}

\citebrackets[]

\begin{document}
    \subfile{sections/cover.tex}

    \section{\MakeUppercase{Introdução}}
    \subfile{sections/introduction.tex}

    \section{\MakeUppercase{O robô quadrúpede}}
    \subfile{sections/theoretical_reference.tex}

    \section{\MakeUppercase{Metodologia}} 
    \subfile{sections/methodology.tex}

    \section{\MakeUppercase{O robô Caramelo}} 
    \subfile{sections/development.tex}

    \section{\MakeUppercase{Resultados e Discussão}}
    \subfile{sections/results.tex}

    \section{\MakeUppercase{Conclusão}}
    \subfile{sections/conclusion.tex}

    
    \bibliography{refs.bib}

\end{document}
